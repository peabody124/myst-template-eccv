\documentclass[runningheads]{llncs}

%%%%%%%%%%%%%%%  Packages   %%%%%%%%%%%%%%%
[-IMPORTS-]

% ---------------------------------------------------------------
% Include basic ECCV package
 
% TODO REVIEW: Insert your submission number below by replacing '*****'
% TODO FINAL: Comment out the following line for the camera-ready version
\usepackage[review,year=2024,ID=*****]{eccv}
% TODO FINAL: Un-comment the following line for the camera-ready version
% \usepackage{eccv}

% OPTIONAL: Un-comment the following line for a version which is easier to read
% on small portrait-orientation screens (e.g., mobile phones, or beside other windows)
%\usepackage[mobile]{eccv}

% ---------------------------------------------------------------
% Other packages

% % Commonly used abbreviations (\eg, \ie, \etc, \cf, \etal, etc.)
\usepackage{eccvabbrv}

% % Include other packages here, before hyperref.
\usepackage{graphicx}
\usepackage{booktabs}

% % The "axessiblity" package can be found at: https://ctan.org/pkg/axessibility?lang=en
% Myst comment: unfortunately this breaks the build, so need to uncomment this and do a 
% final compile in standard tex
% \usepackage[accsupp]{axessibility}  % Improves PDF readability for those with disabilities.


% ---------------------------------------------------------------
% Hyperref package

% It is strongly recommended to use hyperref, especially for the review version.
% Please disable hyperref *only* if you encounter grave issues.
% hyperref with option pagebackref eases the reviewers' job, but should be disabled for the final version.
%
% If you comment hyperref and then uncomment it, you should delete
% main.aux before re-running LaTeX.
% (Or just hit 'q' on the first LaTeX run, let it finish, and you
%  should be clear).

% TODO FINAL: Comment out the following line for the camera-ready version
\usepackage[pagebackref,breaklinks,colorlinks,citecolor=eccvblue]{hyperref}
% TODO FINAL: Un-comment the following line for the camera-ready version
%\usepackage{hyperref}

% Support for ORCID icon
\usepackage{orcidlink}

\usepackage{amsfonts}
\usepackage[version=4]{mhchem} % For chemical notation
\usepackage{siunitx}    % For SI units
\usepackage{pdflscape}  % For putting pages in landscape mode
\usepackage{rotating}   % For rotating specific elements
\usepackage{textgreek}  % Greek symbols
\usepackage{gensymb}    % Symbols
\usepackage[misc]{ifsym} % For the \Letter symbol
\usepackage{orcidlink}  % For the \orcidlink
\usepackage{listings}   % For inserting code chunks
\usepackage{colortbl}   % For Knitr table colouring
\usepackage{tabularx}   % For making Knitr tables compatible
\usepackage{longtable}  % For multi-page tables
\usepackage{subcaption}
\usepackage{multirow}
\usepackage{snotez}     % For sidenote environments. enotez for endnotes
\usepackage{csquotes}   % For language-based quote rules (helps BiBLaTeX)
\usepackage{booktabs}
\usepackage{unicode-math}
\renewcommand{\floatpagefraction}{.8}

%%%%%%%%%%%%%%%  Bibliography   %%%%%%%%%%%%%%%

% Setting up references. Compared to the original template, this is 
% using the splncsnat style which supports natbib references. Also,
% it isn't clear to me why the bibsection needs to be redefined, but
% this makes it consistent with the examples.
\usepackage[numbers,square]{natbib}
\bibliographystyle{splncsnat}
\renewcommand\bibsection{
   \section*{References}
   \markboth{\MakeUppercase{\refname}}{\MakeUppercase{\refname}}
 }


\begin{document}

%%%%%%%%%%%%%%%%   Title   %%%%%%%%%%%%%%%%
\title{[-doc.title-]}


%%%%%%%%%%%%%%%  Author list  %%%%%%%%%%%%%%%

\author{
[# for author in doc.authors #]
	[--author.name-]
	\inst{[--author.affiliations|join(', ', 'index')--]}
	[#- if author.orcid -#]
	%\orcidID{[-author.orcid-]}
	\href{https://orcid.org/[-author.orcid-]}{\orcidicon}
	[#-if author.email-#]
	\href{[-author.email-]}{\Letter}
	[#-endif-#]	
	[#- endif -#]
	[#- if not loop.last #] \and [# endif #]
[# endfor #]
}

\institute{
[# for affiliation in doc.affiliations #]
{[-affiliation.value-]}
[#- if not loop.last #] \and [# endif #]
[# endfor #]
}

% if the length of doc.authors is > 2 use author running with just first author et al
% TODO: need to also parse the name and only show the last word
[# if doc.authors.length > 2 #]
\authorrunning{[-doc.authors[0].name-] et al.}
[# else #]
[# endif #]

\maketitle

[# if parts.abstract #]
\begin{abstract}
[-parts.abstract-]\\

[# if doc.keywords #]
\keywords{[-doc.keywords|join(", ")-]}
[# endif #]

\end{abstract}
[# endif #]


%%%%%%%%%%%%%%%  Main text   %%%%%%%%%%%%%%%

[-CONTENT-]

[# if parts.acknowledgments #]
%%%%%%%%%%%%% Acknowledgements %%%%%%%%%%%%%
\section*{Acknowledgements}
\footnotesize
[- parts.acknowledgments -]
\normalsize
[# endif #]

[# if options.link #]
\section*{Original article}
\footnotesize
This article is available online at the following URL: \href{[-options.link-]}{[-options.link-]}
\normalsize
[# endif #]

%%%%%%%%%%%%%%   Bibliography   %%%%%%%%%%%%%%
[# if doc.bibliography #]
\bibliography{[- doc.bibliography | join(", ") -]}
[# endif #]


%%%%%%%%%%%%%%   Appendix   %%%%%%%%%%%%%%

%% MyST Note: Appendix isn't a part of the original template but is
%% useful.
[# if parts.appendix #]
\clearpage

%%\section*{Appendix}
[-parts.appendix-]
[# endif #]

\end{document}
